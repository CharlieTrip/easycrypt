
\subsection{Trivial judgements: the \rawec{exfalso} tactic}

\subsubsection*{The \rawec{exfalso} tactic}
\index{commonhl}{\rawec{exfalso}}

The \rawec{exfalso} tactic discharges a Hoare, Probabilistic Hoare or
Probabilistic relational Hoare judgement if the precondition can be
trivially proved false. The \rawec{trivial} tactic incorporates the
functionality of the \rawec{exfalso} tactic.


\subsubsection*{The \rawec{trivial} tactic}
\index{commonhl}{\rawec{trivial}}

Discharges trivial judgements by applying the \rawec{exfalso} and
\rawec{pr_bounded} tactics, removing Hoare judgements with
postconditions equivalent to true, modulo trivial reasoning
in the ambient logic.


\subsection{Probability expressions: the \rawec{pr_false, pr_or} tactics}
\index{commonhl}{\rawec{pr_false,pr_or}}

\begin{displaymath}
\infrule{
  \post \Rightarrow \false
}{
  \Prm{c}{m}{\post} = 0
}
\end{displaymath}

\begin{displaymath}
\infrule{
\Prm{c}{m}{\pre} +  \Prm{c}{m}{\post} - \Prm{c}{m}{\pre \wedge \post} = \delta
}{
  \Prm{c}{m}{\pre \vee \post} = \delta
}
\end{displaymath}



\subsection{Program transformations}
%

\subsubsection*{The \rawec{inline} tactic}
\index{commonhl}{\rawec{inline}}
%

\subsubsection*{The \rawec{swap} tactic}
\index{commonhl}{\rawec{swap}}
%
\Syntax \rawec{swap} [\textit{side}] \textit{swap\_pos}

\textbf{where:} 
\begin{tabular}[t]{l}
  \textit{swap\_pos} ::= 
  \textit{n} \textit{n} \textit{n} $\mid$ \textit{n} \textit{z} $\mid$ [\textit{n}:\textit{n}] \textit{z}
  \\
  $n$ a natural number
  \\
  $z$ an integer number
\end{tabular}
  

The tactic [\rawec{swap} $p_1$ $p_2$ $p_3$] swaps the code between
positions $p_1$ and $p_2$ with the code between positions $p_2$ and
$p_3$. That is, assuming that $c_1$ and $c_2$ are syntactically
independent, that $c_1$ is between positions $p_1$ and $p_2$ and that
$c_2$ is between positions $p_2$ and $p_3$, the tactic implements the
following rule:
\begin{displaymath}
\infrule{
  \Hoare{c;c_2;c_1;c_3}{\pre}{\post}
}{
  \Hoare{c;c_1;c_2;c_3}{\pre}{\post}
} [\mathec{swap}\ p_1\ p_2\ p_3]
\end{displaymath}

If $k$ is positive (negative) then [\rawec{swap} $k$] moves the first
(last) instruction $k$ positions forwards (backwards). Similarly,
[\rawec{swap} $i$ $k$] moves the $i^{th}$ instruction forwards or
backwards, and [\rawec{swap} $[i_1:i_2]$ $k$] moves the instructions
between positions $i_1$ and $i_2$.

\subsubsection*{The \rawec{fun} tactic}
\index{commonhl}{\rawec{fun}}
\NotDocumented
\subsubsection*{The \rawec{unroll} tactic}
\index{commonhl}{\rawec{unroll}}
\NotDocumented
\subsubsection*{The \rawec{splitwhile} tactic}
\index{commonhl}{\rawec{splitwhile}}
\NotDocumented
\subsubsection*{The \rawec{fusion,fission} tactic}
\index{commonhl}{\rawec{fusion,fission}}
\NotDocumented
\subsubsection*{The \rawec{condt,condf} tactic}
\index{commonhl}{\rawec{condt,condf}}
\NotDocumented
\subsubsection*{The \rawec{kill} tactic}
\index{commonhl}{\rawec{kill}}
\NotDocumented


%%% Local Variables: 
%%% mode: latex
%%% TeX-master: "easycrypt"
%%% End: 
